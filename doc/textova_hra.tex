\documentclass[main.tex]{subfiles}

\begin{document}


\subsection{Textové hry}
Textová hra je počítačová hra, která pro interakci s uživatelem používá text, přesněji řečeno, množinu standardně kódovaných znaků, historicky například pomocí acii. Jiným typem počítačových her, dnes převažujících, jsou videohry, které naopak využívají vektorovou nebo bitmapovou grafiku. \cite{web:wik:en:textgame} Alternativně lze v žánru textových her doplnit text obrázky, zvuky, případně jinými vjemy. Charakterizující prvek textových her ale zůstává slovní popis herního prostředí. V některých připadech, tzv. grafických textových her, rozhoduje hráč o vývoji hry kliknutím na jedno z nabízených tlačítek.\cite{web:wik:cz:textovahra}

\subsubsection{Textové hry vs. adventury}
Základním znakem adventur je, že hráč se snaží splnit úkoly, které mu hra zadá. Není podmínkou textový režim - existuje velké množství grafických adventur. \cite{web:wik:cz:adventura}

\subsubsection{Historie}
První zdokumentované textové hry se začaly objevovat v 60. letech minulého století. Byly provozovány na sálových počítačích, tzv. mainframech s přídavným výstupem realizovaným pomocí dálnopisu, který tiskl výstup hry na papír. V 60. letech sice již existovaly videoterminály, ale jejich větší rozšíření bylo omezeno vysokou cenou. \cite{web:pcmag:en:oldcomputergames, web:wik:en:textgame}


Příklady takových her jsou The Oregon Trail (1971), Lunar Lander (1969), Dungeon (1977). \cite{web:pcmag:en:oldcomputergames}

V sedmdesátých letech videoterminály nahradily dálnopisy v rozhranní sálových počítačů (mainfraimů).Stalo se tak díky zlevňování technologií. Dalšímu rozvoji textových her napomohl rozmach osobních počítačů. Většina her té doby byla inspirována hrou Dungeons \& Dragons nebo světem v dílech J.R.R.Tolkiena. \cite{web:pcgamer:en:historyrpgs}  


\subsubsection{České textové hry}
Jedním z nejznámějších českých vývojářů textových her a adventur je František Fuka, člen programátorské skupiny Golden Triangle, která vyvinula množství textových her. Fuka je spoluautorem například her Belegost, Podraz 3, Indiana Jones.

Textovým hrám se věnuje množství webových stránek. Archivem českých a slovenských textových her je například webová stránka \href{https://www.textovky.cz}{textovky.cz}, kde je možné zahrát si historické textové hry a adventury, převážně českých a slovenských autorů. \cite{web:text:cz:archiv}

\end{document}
