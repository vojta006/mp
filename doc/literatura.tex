
\documentclass[main.tex]{subfiles}
\begin{document}



\renewcommand{\refname}{Seznam použité literatury a~zdrojů informací} 
\phantomsection %pridej odkaz do PDF zalozek

\begin{thebibliography}{99}

%textovky

%\bibitem{web:pcgamer:en:history-of-rpgs} The history of RPGs [online]. May 06, 2021 [cit. 2023-01-04]. Dostupné z: \href{https://www.pcgamer.com/the-complete-history-of-rpgs}
%\bibitem{web:text:cz:archiv} Textovkycz [online]. [cit. 2023-01-04]. Dostupné z: \href{https://www.textovky.cz}
%\bibitem{web:pcmag:en:old-computer-games} The Forgotten World of Teletype Computer Games [online]. [cit. 2023-01-04]. Dostupné z: \url{https://www.pcmag.com/news/the-forgotten-world-of-teletype-computer-games}
%\bibitem{web:wik:en:textgame} Text-based game. In: Wikipedia: the free encyclopedia [online]. San Francisco (CA): Wikimedia Foundation, 2001- [cit. 2023-01-04]. Dostupné z: \url{https://en.wikipedia.org/wiki/Text-based-game}
%HTML zdroje
\bibitem{web:en:htmlhistory} HISTORY OF HTML [online]. [cit. 2023-01-06]. Dostupné z: \url{https://www.bu.edu/lernet/artemis/years/2020/projects/FinalPresentations/HTML/historyofhtml.html} 
\bibitem{web:wik:en:html} HTML. In: Wikipedia: the free encyclopedia [online]. San Francisco (CA): Wikimedia Foundation, 2001- [cit. 2023-01-06]. Dostupné z: \url{https://en.wikipedia.org/wiki/HTML}
\bibitem{web:en:countinghtml} On the Difficulty of Counting the Number of HTML Elements [online]. [cit. 2023-01-06]. Dostupné z: \url{https://meiert.com/en/blog/the-number-of-html-elements/}
\bibitem{web:wik:cz:html} Hypertext\_Markup\_Language. In: Wikipedia: the free encyclopedia [online]. San Francisco (CA): Wikimedia Foundation, 2001- [cit. 2023-01-06]. Dostupné z: \url{https://cs.wikipedia.org/wiki/Hypertext\_Markup\_Language}
%CSS zdroje

\bibitem{web:cz:selektory} CSS2 – selektory, pseudotřídy a pseudoelementy [online]. [cit. 2023-01-06]. Dostupné z: \url{https://www.interval.cz/clanky/css2-selektory-pseudotridy-a-pseudoelementy/}

%javascript
\bibitem{web:wik:cz:js} JavaScript. In: Wikipedia: the free encyclopedia [online]. San Francisco (CA): Wikimedia Foundation, 2001- [cit. 2023-01-06]. Dostupné z: \url{https://cs.wikipedia.org/wiki/JavaScript}
\bibitem{web:wik:en:js} JavaScript. In: Wikipedia: the free encyclopedia [online]. San Francisco (CA): Wikimedia Foundation, 2001- [cit. 2023-01-06]. Dostupné z: \url{https://en.wikipedia.org/wiki/JavaScript}

%oop
\bibitem{web:wik:en:oop}Object-oriented programming. In: Wikipedia: the free encyclopedia [online]. San Francisco (CA): Wikimedia Foundation, 2001- [cit. 2023-01-06]. Dostupné z: \url{https://en.wikipedia.org/wiki/Object-oriented\_programming}
%\bibitem{web:wik:cz:oop}Objektově orientované programování. In: Wikipedia: the free encyclopedia [online]. San Francisco (CA): Wikimedia Foundation, 2001- [cit. 2023-01-06]. Dostupné z: \url{https://cs.wikipedia.org/wiki/Objektov%C4%9B_orientovan%C3%A9_programov%C3%A1n%C3%AD}

%git 
\bibitem{web:wik:en:git} Git. In: Wikipedia: the free encyclopedia [online]. San Francisco (CA): Wikimedia Foundation, 2001- [cit. 2022-12-18].\\ Dostupné z: \url{https://cs.wikipedia.org/wiki/Git}
\bibitem{web:git\_scm:en:man} Reference [online]. [cit. 2023-01-06]. Dostupné z: \url{https://git-scm.com/docs}
%tex
\bibitem{web:wik:en:tex} Tex. In: Wikipedia: the free encyclopedia [online]. San Francisco (CA): Wikimedia Foundation, 2001- [cit. 2022-12-19]. Dostupné z: \url{https://en.wikipedia.org/wiki/TeX}

\bibitem{web:wik:en:latex} Tex. In: Wikipedia: the free encyclopedia [online]. San Francisco (CA): Wikimedia Foundation, 2001- [cit. 2022-12-19]. Dostupné z: \url{https://en.wikipedia.org/wiki/LaTeX}

\end{thebibliography}



\end{document}
