
\documentclass[main.tex]{subfiles}
\begin{document}

\section{Závěr}
Byla vytvořena funkční textová hra s inspirací tématem zkáza raketoplánu Columbia. Hra se odehrává na palubě fiktivní rakety Shumaker-Levi 9 a ve volném vesmíru, a nabízí hráči kolem 20 příkazů, z nichž některé popisují stejné děje. Hrací čas je velmi rozdílný podle zkušenosti hráče s textovými, či jinými hrami. Obecně lze říci, že by mělo být běžné dohrát hru bez předchozí zkušenosti do jedné hodiny. 

Na počátku bylo zamýšleno vytvořit funkcionalitu, ve které by bylo bývalo nutné pilníkem upilovat bit aku vrtačky, který by býval byl nekompatibilní. To však nebylo realizováno, neboť by bývala mohla být zhoršena hratelnost.

\subsection{Potíže}
Jelikož není javascript primárně určen k psaní textových her či adventur, neexistují příliš dobré nástroje, které by jejich vývoj usnadňovaly. Nelze říci, že by se během vývoje hry vyskytly vážnější problémy, spíše problémy, které byly způsobeny neoptimalitou javascriptu jako jazyka pro vývoj textových her, jako například délka kódu.

\subsection{Rozšíření} 
Pro snazší vývoj by mohl být použit programovací jazyk TADS 3. Dále by samotná hra mohla být libovolným způsobem rozšiřována přidáváním hracích místností, objektů, funkcí, tlačítek, ... Plánovanou funkcionalitou byla nutnost upilovat
\end{document}
