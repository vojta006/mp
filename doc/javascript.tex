
\documentclass[main.tex]{subfiles}
\begin{document}

\section{Javascript}
Javascript je programovací jazyk, který je společně s HTML a CSS jednou z hlavních technologií používaných pro tvorbu webových stránek, nicméně jeho použití se neomezuje pouze na webové stránky. U webových stránek se používá pro zajištění interaktivního designu a interakci s uživatelem. V roce 2022 98\% webových stránek používalo javascript na straně klienta. Syntaxe Javascriptu patří do rodiny jazyků C/C++/Java. Navzdory podobnému jménu jsou Javascript a Java odlišné jazyky a název Java je v případě Javascriptu zahrnut pouze z marketingových důvodů.


\subsection{Vlastnosti}
Na anglické wikipedii jsou vyjmenovány následující vlastnosti:
\begin{quote} \textit{" JavaScript is a high-level, often just-in-time compiled language that conforms to the ECMAScript standard. It has dynamic typing, prototype-based object-orientation, and first-class functions. It is multi-paradigm, supporting event-driven, functional, and imperative programming styles. It has application programming interfaces (APIs) for working with text, dates, regular expressions, standard data structures, and the Document Object Model (DOM)." } \end{quote}

Nyní budou některé vlastnosti vysvětleny. Objektově orientovanému programování je věnována samostatná kapitola \nameref{oop}.
\begin{itemize}
    \item Hight-level - (vysokoúrovňový programovací jazyk). Tyto jazyky disponují silnou abstrakcí v tom, jak je zacházeno s hardwarem počítače. Opakem jsou nízkoúrovňové jazyky, jež jsou naopak s hardwarem provázány velmi úzce a je například možné přistupovat k jednotlivým adresám v paměti počítače.
    \item just-in-time compiled (JIT) - (zřejmě neexistuje český překlad) je způsob, kdy nejsou zdrojové kódy překládány do strojového kódu před spuštěním programu, nýbrž až za běhu programu, často z takzvaného bytekódu, již kompilovaného zdrojového kódu. Obvykle dosahuje většího výkonu než interpretované jazyky a zároveň kombinuje jejich výhody (např. optimalizace možné až za běhu programu).
    \item dynamic typing - (dynamicky typovaný). Vlastnost programovacích jazyků, kdy je typ proměnné vázán na hodnotu. Není tedy potřeba při deklaraci proměnné definovat její typ - ten je určen podle proměnné.
    \item first-class functions - 
\end{itemize}

\subsection{Historie}
Do roku 1995 byly webové prohlížeče schopné zobrazovat pouze statický obsah, bez možnosti jakkoli zobrazit pohyblivé prvky. To se rozhodla změnit společnost Netscape, která se začala ubírat dvěma cestami. Ve spolupráci se společností Sun Microsystem začala pracovat na zakomponování existujícího jazyka Java do jejich webového prohlížeče Navigator a také pověřila Brendana Eicha zakomponováním jazyku Scheme do Navigatoru. Nadřízení Eicha trvali na tom, aby syntaxe jazyka připomínala Javu, proto Eich upustil od zakomponování Scheme do prohlížeče a místo toho napsal během 10 dnů vlastní jazyk. Nejprve pojmenován Mocha, v září 1995 přejmenován na LiveScript, a nakonec v prosinci 1995, při oficiálním vydání, přejmenován na JavaScript. Souběžně také vyvinul pro JavaScript interpretr.

V roce 1995 vznikl také prohlížeč Internet Explorer, jenž byl vyvinut společnostj Microsoft. Reverzním inženýrstvím Navigatoru vyvinul do Exploreru vlastní skriptovací jazyk JScript, jež byl s JavaScriptem pouze částečně kompatibilní. To představovalo značné nesnáze pro webové vývojáře, kteří museli dělat kompromisy, aby webové stránky vypadaly obstojně v obou majoritních prohlížečích - Internet Exploreru i Navigatoru. 

Tato dualita ve skriptovacích jazycích pokračovala až do roku 2004. Během té doby rostla popularita Exploreru až tak, že v roce 2000 zaujímal 95\% podíl na trhu, což v podstatě učinilo JScript standardním jazykem webového programování. V roce 2004 vyšel pod organizací Mozilla, následovníkem Nescapu, open-source prohlížeč Firefox. 
Opětovnému nárůstu popularity JavaScriptu pomohl Jessie James Garret, když publikoval článek o nových webových technologiích založených na JavaScriptu a hlavně jejich souhrnnou filozofii - Ajax (Asynchronous JavaScript and XML). Ten popisoval asynchronní řízení událostí, například načítání dat na pozadí, což se pozitivně projevilo například v rychlosti načítání stránek. Na tento rozvoj navázalo mnoho open-source knihoven zjednodušujících psaní webových aplikací jako Prototype(2005) nebo JQuery(2006).
Rozšířit JavaScript mimo prohlížeč se povedlo také díky Ryanu Dahlovi, který napsal prostředí Node.js (2009), které umožňuje spouštět JavaScript, například na straně serveru.

V současnosti je JavaScript jediným skriptovacím jazykem na straně uživatele, který je webovými prohlížeči podporován, což z něj dělá jeden z nejrozšířenějších programovacích jazyků.

%https://www.quora.com/Why-does-Javascript-seem-so-weird


\end{document}
