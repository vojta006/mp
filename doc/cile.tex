\documentclass[main.tex]{subfiles}

\begin{document}

\section{Úvod}
Textové hry jsou jedním z prvních odvětví počítačových her, které se začalo rozvíjet s rozšířením osobních počítačů v sedmdesátých letech minulého století. Jedím z důvodů takto brzkého rozmachu jsou nízké nároky na výpočetní výkon a relativní nenáročnost vývoje oproti ostatním počítačovým hrám, videohrám. Proto byl jejich provoz možný i na velmi málo výkonných sálových počítačích bez videokonzole.Později na osobních počítačích, již s videokonzolemi. \cite{web:pcmag:en:oldcomputergames}
 Od té doby je téměř kompletně nahradilo odvětví videoher. Přesto se však v dnešní době věnuje vývoji textových her řada nadšenců, primárně na nekomerční úrovni, kteří spoléhají na hlavní kouzlo textových her, tedy lidskou představivost. \cite{web:wik:en:textgame}

Pro toto téma jsem se rozhodl, protože se zajímám o informatiku a programování, a chtěl jsem se blíže seznámit a zdokonalit ve vývoji webových stránek. Jejich podrobná znalost je při téměř jakémkoli komerčním vývoji softwaru esenciální.

\subsection{Cíle}
Cílem této práce je vytvořit textovou hru, provozovanou ve webovém prohlížeči, s využitím dnes nejčastěji používaných webových technologií. Zejména se jedná o HTML, CSS, javascript, jejichž podrobnému popisu budou věnovány následující kapitoly. K realizaci však budou nepřímo využity i jiné nástroje, jako verzovací program git a \LaTeX, kterým se také budeme věnovat.



\end{document}
