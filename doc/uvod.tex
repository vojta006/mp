
\documentclass[main.tex]{subfiles}
\begin{document}


\subsection*{Poděkování}
Děkuji vedoucímu Mgr. Janu Souhradovi za vedení, pomoc a cenné rady při realizaci Maturitní práce. 

\subsection*{Čestné prohlášení}
Čestně prohlašuji, že jsem tuto práci vypracoval samostatně s použitím literatury a zdrojů uvedených v seznamu použité literatury.

\subsection*{Abstrakt}
Tato práce se zabývá návrhem a realizací textové hry napsané za použití webových nátrojů - HTML, CSS a JavaScriptu. Hra byla inspirována zkázou raketoplánu Columbia, která se odehrála v roce 2003. HTML a CSS byly použity pro grafický vzhled statického webového dokumentu, zatímco JavaScript je použit jako hlaví programovací jazyk. Zpracovává příkazy od uživatele a jejich základě dynamicky upravuje dokument. Dále se v teoretické části pojednává verzovacích systémech a značkovacím jazyku \LaTeX, neboť byly využity při tvorbě práce.
Byla vytvořena textová hra, odehrávající se na palubě fiktivní rakety Shumaker-Levi 9. Hra se odehrává ve 4 místnostech a je koncipována do jedné hodiny hraní. 

\subsection*{Klíčová slova}
HTML, CSS, JavaScript, textová hra
\end{document}
