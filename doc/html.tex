\documentclass[main.tex]{subfiles}
\begin{document}


\section{HTML}
Hypertext Markup Language je značkovací jazyk používaný pro tvorbu statických webových stránek. Spolu s CSS a javascriptem patří k základním technologiím pouřívaným při vývoji webové stránky. 
\subsection{Historie}
Autorem jazyka je Tim Berners-Lee, který jej publikoval v roce 1990 za účelem usnadnění publikování vědeckých článků, na které se obvykle používaly složitější jazyky Tex a postscript. Od té doby vzniklo 5 verzí jazyka, zatím poslední, HTML 5.2, byla vydána v prosinci 2017.

\subsection{Specifikace}
"Jazyk HTML je charakterizován množinou značek (tzv. tagů) a jejich vlastností (atributů) definovaných pro danou verzi." citace Tagy se rodělují na párové a nepárové a uzavírají se do špičatých závorek.


\end{document}

